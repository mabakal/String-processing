% Options for packages loaded elsewhere
\PassOptionsToPackage{unicode}{hyperref}
\PassOptionsToPackage{hyphens}{url}
%
\documentclass[
]{article}
\usepackage{amsmath,amssymb}
\usepackage{iftex}
\ifPDFTeX
  \usepackage[T1]{fontenc}
  \usepackage[utf8]{inputenc}
  \usepackage{textcomp} % provide euro and other symbols
\else % if luatex or xetex
  \usepackage{unicode-math} % this also loads fontspec
  \defaultfontfeatures{Scale=MatchLowercase}
  \defaultfontfeatures[\rmfamily]{Ligatures=TeX,Scale=1}
\fi
\usepackage{lmodern}
\ifPDFTeX\else
  % xetex/luatex font selection
\fi
% Use upquote if available, for straight quotes in verbatim environments
\IfFileExists{upquote.sty}{\usepackage{upquote}}{}
\IfFileExists{microtype.sty}{% use microtype if available
  \usepackage[]{microtype}
  \UseMicrotypeSet[protrusion]{basicmath} % disable protrusion for tt fonts
}{}
\makeatletter
\@ifundefined{KOMAClassName}{% if non-KOMA class
  \IfFileExists{parskip.sty}{%
    \usepackage{parskip}
  }{% else
    \setlength{\parindent}{0pt}
    \setlength{\parskip}{6pt plus 2pt minus 1pt}}
}{% if KOMA class
  \KOMAoptions{parskip=half}}
\makeatother
\usepackage{xcolor}
\usepackage[margin=1in]{geometry}
\usepackage{color}
\usepackage{fancyvrb}
\newcommand{\VerbBar}{|}
\newcommand{\VERB}{\Verb[commandchars=\\\{\}]}
\DefineVerbatimEnvironment{Highlighting}{Verbatim}{commandchars=\\\{\}}
% Add ',fontsize=\small' for more characters per line
\usepackage{framed}
\definecolor{shadecolor}{RGB}{248,248,248}
\newenvironment{Shaded}{\begin{snugshade}}{\end{snugshade}}
\newcommand{\AlertTok}[1]{\textcolor[rgb]{0.94,0.16,0.16}{#1}}
\newcommand{\AnnotationTok}[1]{\textcolor[rgb]{0.56,0.35,0.01}{\textbf{\textit{#1}}}}
\newcommand{\AttributeTok}[1]{\textcolor[rgb]{0.13,0.29,0.53}{#1}}
\newcommand{\BaseNTok}[1]{\textcolor[rgb]{0.00,0.00,0.81}{#1}}
\newcommand{\BuiltInTok}[1]{#1}
\newcommand{\CharTok}[1]{\textcolor[rgb]{0.31,0.60,0.02}{#1}}
\newcommand{\CommentTok}[1]{\textcolor[rgb]{0.56,0.35,0.01}{\textit{#1}}}
\newcommand{\CommentVarTok}[1]{\textcolor[rgb]{0.56,0.35,0.01}{\textbf{\textit{#1}}}}
\newcommand{\ConstantTok}[1]{\textcolor[rgb]{0.56,0.35,0.01}{#1}}
\newcommand{\ControlFlowTok}[1]{\textcolor[rgb]{0.13,0.29,0.53}{\textbf{#1}}}
\newcommand{\DataTypeTok}[1]{\textcolor[rgb]{0.13,0.29,0.53}{#1}}
\newcommand{\DecValTok}[1]{\textcolor[rgb]{0.00,0.00,0.81}{#1}}
\newcommand{\DocumentationTok}[1]{\textcolor[rgb]{0.56,0.35,0.01}{\textbf{\textit{#1}}}}
\newcommand{\ErrorTok}[1]{\textcolor[rgb]{0.64,0.00,0.00}{\textbf{#1}}}
\newcommand{\ExtensionTok}[1]{#1}
\newcommand{\FloatTok}[1]{\textcolor[rgb]{0.00,0.00,0.81}{#1}}
\newcommand{\FunctionTok}[1]{\textcolor[rgb]{0.13,0.29,0.53}{\textbf{#1}}}
\newcommand{\ImportTok}[1]{#1}
\newcommand{\InformationTok}[1]{\textcolor[rgb]{0.56,0.35,0.01}{\textbf{\textit{#1}}}}
\newcommand{\KeywordTok}[1]{\textcolor[rgb]{0.13,0.29,0.53}{\textbf{#1}}}
\newcommand{\NormalTok}[1]{#1}
\newcommand{\OperatorTok}[1]{\textcolor[rgb]{0.81,0.36,0.00}{\textbf{#1}}}
\newcommand{\OtherTok}[1]{\textcolor[rgb]{0.56,0.35,0.01}{#1}}
\newcommand{\PreprocessorTok}[1]{\textcolor[rgb]{0.56,0.35,0.01}{\textit{#1}}}
\newcommand{\RegionMarkerTok}[1]{#1}
\newcommand{\SpecialCharTok}[1]{\textcolor[rgb]{0.81,0.36,0.00}{\textbf{#1}}}
\newcommand{\SpecialStringTok}[1]{\textcolor[rgb]{0.31,0.60,0.02}{#1}}
\newcommand{\StringTok}[1]{\textcolor[rgb]{0.31,0.60,0.02}{#1}}
\newcommand{\VariableTok}[1]{\textcolor[rgb]{0.00,0.00,0.00}{#1}}
\newcommand{\VerbatimStringTok}[1]{\textcolor[rgb]{0.31,0.60,0.02}{#1}}
\newcommand{\WarningTok}[1]{\textcolor[rgb]{0.56,0.35,0.01}{\textbf{\textit{#1}}}}
\usepackage{graphicx}
\makeatletter
\def\maxwidth{\ifdim\Gin@nat@width>\linewidth\linewidth\else\Gin@nat@width\fi}
\def\maxheight{\ifdim\Gin@nat@height>\textheight\textheight\else\Gin@nat@height\fi}
\makeatother
% Scale images if necessary, so that they will not overflow the page
% margins by default, and it is still possible to overwrite the defaults
% using explicit options in \includegraphics[width, height, ...]{}
\setkeys{Gin}{width=\maxwidth,height=\maxheight,keepaspectratio}
% Set default figure placement to htbp
\makeatletter
\def\fps@figure{htbp}
\makeatother
\setlength{\emergencystretch}{3em} % prevent overfull lines
\providecommand{\tightlist}{%
  \setlength{\itemsep}{0pt}\setlength{\parskip}{0pt}}
\setcounter{secnumdepth}{-\maxdimen} % remove section numbering
\ifLuaTeX
  \usepackage{selnolig}  % disable illegal ligatures
\fi
\IfFileExists{bookmark.sty}{\usepackage{bookmark}}{\usepackage{hyperref}}
\IfFileExists{xurl.sty}{\usepackage{xurl}}{} % add URL line breaks if available
\urlstyle{same}
\hypersetup{
  pdftitle={R Notebook},
  hidelinks,
  pdfcreator={LaTeX via pandoc}}

\title{R Notebook}
\author{}
\date{\vspace{-2.5em}}

\begin{document}
\maketitle

\hypertarget{string-processing}{%
\section{String processing}\label{string-processing}}

\hypertarget{introduction}{%
\subsubsection{Introduction}\label{introduction}}

Tous tâche d'un data scientist commence par la mis á propres des données
á analyser. Le Data cleaning est un étape indispensable dans l'analyse
de données et il prends la grande partie d'un de temps d'un scientifique
de données. Il peux consister au traitement des chaînes de charactères,
des valeurs maquantes, convertion de types, opérations sur les dates
etc. Il est très cruciale de bien purifier les données car une mauvaise
purification de données peut entrainer, Une erreurs d'analyse, une perte
d'information ou encore une prise de décision erroné, etc.

Le traitement avec les chaînes de caractère est l'une opération dans la
purification de données, dans ce cas d'utilisation on va voir les
différents problèmes qu'on peut rencontrer quand on traite avec les
caractères on verras plus précisement:

\begin{itemize}
\item
  Extractions des nombres d'une chaines de caractères
\item
  Suppressions des charactères non désiré d'un text
\item
  recherche d'un modèle spécifique et le remplacer par un autre dans une
  chaine
\item
  extraction une partie spécifique d'un text
\item
  Division d'une chaine de caractère en de multiples parties
\end{itemize}

Le packages nommés \textbf{base} de R fournie des fonctions pour
éffectué toutes ces tâches cependant le nom de ces fonction n'est pas
faciles á mémoriser. Heureusement il y a un package de R appélé
\textbf{stringr} qui réalise d'une manière concise l'ensemble de ces
opérations mais aussi le nom est faciles a mémoriser, toutes les
fonctions opérant sur les chaines de charactères commence par le mot
\textbf{str\_}. Par exemple la fonction \textbf{str\_split} de stringr
est la fonction qui divise une chaine de charactère selon un séparateur
qui lui est fournie.

\hypertarget{importation-des-bibliothuxe8ques}{%
\paragraph{Importation des
bibliothèques}\label{importation-des-bibliothuxe8ques}}

\begin{itemize}
\item
  \textbf{tidyverse} une groupe de packages regroupant tous les outils
  pour l'analyse de données
\item
  \textbf{stringr} le package en question qui fournie les outils pour
  traiter avec les chaines de charactères
\item
  \textbf{Dslabs} est un packages construit pour les data scientists, il
  contient les fonctions et les datasets utilisés par pour pratiquer
  l'ensemble des challenges qu'on peut rencontrer en analyse de données.
\end{itemize}

\begin{Shaded}
\begin{Highlighting}[]
\FunctionTok{library}\NormalTok{(tidyverse)}
\end{Highlighting}
\end{Shaded}

\begin{verbatim}
## -- Attaching core tidyverse packages ------------------------ tidyverse 2.0.0 --
## v dplyr     1.1.2     v readr     2.1.4
## v forcats   1.0.0     v stringr   1.5.0
## v ggplot2   3.4.2     v tibble    3.2.1
## v lubridate 1.9.2     v tidyr     1.3.0
## v purrr     1.0.1     
## -- Conflicts ------------------------------------------ tidyverse_conflicts() --
## x dplyr::filter() masks stats::filter()
## x dplyr::lag()    masks stats::lag()
## i Use the conflicted package (<http://conflicted.r-lib.org/>) to force all conflicts to become errors
\end{verbatim}

\begin{Shaded}
\begin{Highlighting}[]
\FunctionTok{library}\NormalTok{(stringr)}
\FunctionTok{library}\NormalTok{(dslabs)}
\end{Highlighting}
\end{Shaded}

dslabs contient un dataset appélé \textbf{reported\_heights}, il est
construits á la suite d'un formulaire web que les etudiant on remplis
leur demandant de reporter leur tailles. Nous allons l'utilisé dans ce
documents

\begin{Shaded}
\begin{Highlighting}[]
\FunctionTok{data}\NormalTok{(}\StringTok{"reported\_heights"}\NormalTok{)}
\end{Highlighting}
\end{Shaded}

\begin{Shaded}
\begin{Highlighting}[]
\FunctionTok{class}\NormalTok{(reported\_heights}\SpecialCharTok{$}\NormalTok{height)}
\end{Highlighting}
\end{Shaded}

\begin{verbatim}
## [1] "character"
\end{verbatim}

Les tailles reportées sont des chaîne de caractères. Créons une variable
x a qui on affecte la taille.

\begin{Shaded}
\begin{Highlighting}[]
\NormalTok{x }\OtherTok{\textless{}{-}}\NormalTok{ reported\_heights}\SpecialCharTok{$}\NormalTok{height}
\end{Highlighting}
\end{Shaded}

\begin{Shaded}
\begin{Highlighting}[]
\NormalTok{x[}\FunctionTok{c}\NormalTok{(}\DecValTok{1}\SpecialCharTok{:}\DecValTok{30}\NormalTok{, }\DecValTok{60}\SpecialCharTok{:}\DecValTok{100}\NormalTok{, }\DecValTok{150}\SpecialCharTok{:}\DecValTok{200}\NormalTok{)]}
\end{Highlighting}
\end{Shaded}

\begin{verbatim}
##   [1] "75"                     "70"                     "68"                    
##   [4] "74"                     "61"                     "65"                    
##   [7] "66"                     "62"                     "66"                    
##  [10] "67"                     "72"                     "6"                     
##  [13] "69"                     "68"                     "69"                    
##  [16] "66"                     "75"                     "64"                    
##  [19] "60"                     "67"                     "66"                    
##  [22] "5' 4\""                 "70"                     "73"                    
##  [25] "72"                     "69"                     "69"                    
##  [28] "72"                     "64"                     "72"                    
##  [31] "67"                     "69"                     "73"                    
##  [34] "74"                     "70"                     "66"                    
##  [37] "511"                    "72"                     "65"                    
##  [40] "65"                     "70"                     "73"                    
##  [43] "67"                     "72"                     "68"                    
##  [46] "68"                     "65"                     "72"                    
##  [49] "71"                     "65"                     "72"                    
##  [52] "69"                     "70"                     "72"                    
##  [55] "6"                      "62"                     "65"                    
##  [58] "70"                     "60"                     "67"                    
##  [61] "62"                     "71"                     "63"                    
##  [64] "68"                     "64.1732"                "64"                    
##  [67] "71"                     "68.5"                   "62"                    
##  [70] "2"                      "70"                     "5'3\""                 
##  [73] "73"                     "68"                     "77"                    
##  [76] "70.5"                   "63"                     "69"                    
##  [79] "69"                     "68.89"                  "66.5"                  
##  [82] "64.173"                 "63"                     "65"                    
##  [85] "64"                     "63"                     "63"                    
##  [88] "69"                     "69"                     "64"                    
##  [91] "62"                     "70"                     "70"                    
##  [94] "59"                     "65"                     "67.7"                  
##  [97] "72"                     "74"                     "71.7"                  
## [100] "70.87"                  "66"                     "72"                    
## [103] "74"                     "69"                     "71"                    
## [106] "70"                     "70"                     "64"                    
## [109] "5 feet and 8.11 inches" "68"                     "66"                    
## [112] "64"                     "67"                     "65"                    
## [115] "72"                     "5.25"                   "70"                    
## [118] "64.57"                  "51"                     "63"                    
## [121] "70"                     "68"
\end{verbatim}

Les données reporté ici, ne sont pas sous le même format même s'il sont
correct, il serait difficile de les analysé, nous allons les ramener
sous le même format. Les étudiant devraient rapporter leur tailles en
inches qui est une unité de mésure utilisé dans certains pays comme le
Royaume Uni, le canada et les USA .Mais il y a certains qui ont reporter
en \textbf{Cm}, d'autre en \textbf{Feet} et d'autre en lettre comme
\textbf{5 feet and 8.11 inches}

Premièrement on récupérer les données qui ne sont pas en
\textbf{pouces}( inches) ni en \textbf{Cm} pour cela on écirt une
fonction qui récupère ces données. La taille minimal en \textbf{inche}
est 50 et le maximal est 80. La fonction prends en paramètre minimun de
inches, le maximum de inches et le données en question

la fonction récupere les valeurs non disponible après être transformé en
numerique, les numérique qui sont inférieur au minimum et supérieur au
maximum ce qui veux dire qu'il ne sont pas en inches, et le dernier
c'est ceux qui ne sont pas noté en centimètre. Pour les autres ils sont
correctement écrit en inches

\begin{Shaded}
\begin{Highlighting}[]
\NormalTok{not\_inches\_or\_cm }\OtherTok{\textless{}{-}} \ControlFlowTok{function}\NormalTok{(x, }\AttributeTok{smallest =} \DecValTok{50}\NormalTok{, }\AttributeTok{tallest =} \DecValTok{84}\NormalTok{)\{ }
\NormalTok{inches }\OtherTok{\textless{}{-}} \FunctionTok{suppressWarnings}\NormalTok{(}\FunctionTok{as.numeric}\NormalTok{(x))}
\NormalTok{ind }\OtherTok{\textless{}{-}} \SpecialCharTok{!}\FunctionTok{is.na}\NormalTok{(inches) }\SpecialCharTok{\&}\NormalTok{ ((inches }\SpecialCharTok{\textgreater{}=}\NormalTok{ smallest }\SpecialCharTok{\&}\NormalTok{ inches }\SpecialCharTok{\textless{}=}\NormalTok{ tallest) }\SpecialCharTok{|}\NormalTok{ (inches}\SpecialCharTok{/}\FloatTok{2.54} \SpecialCharTok{\textgreater{}=}\NormalTok{ smallest }\SpecialCharTok{\&}\NormalTok{ inches}\SpecialCharTok{/}\FloatTok{2.54} \SpecialCharTok{\textless{}=}\NormalTok{ tallest))}
\SpecialCharTok{!}\NormalTok{ind \}}

\NormalTok{problematiques }\OtherTok{\textless{}{-}}\NormalTok{ reported\_heights }\SpecialCharTok{\%\textgreater{}\%} \FunctionTok{filter}\NormalTok{(}\FunctionTok{not\_inches\_or\_cm}\NormalTok{(height)) }\SpecialCharTok{\%\textgreater{}\%} \FunctionTok{pull}\NormalTok{(height)}
\end{Highlighting}
\end{Shaded}

\begin{Shaded}
\begin{Highlighting}[]
\FunctionTok{c}\NormalTok{(}\FunctionTok{length}\NormalTok{(problematiques), }\FunctionTok{length}\NormalTok{(x))}
\end{Highlighting}
\end{Shaded}

\begin{verbatim}
## [1]  200 1095
\end{verbatim}

\begin{Shaded}
\begin{Highlighting}[]
\FunctionTok{as\_tibble}\NormalTok{(problematiques)}
\end{Highlighting}
\end{Shaded}

\begin{verbatim}
## # A tibble: 200 x 1
##    value   
##    <chr>   
##  1 "6"     
##  2 "5' 4\""
##  3 "5.3"   
##  4 "165cm" 
##  5 "511"   
##  6 "6"     
##  7 "2"     
##  8 "5'7"   
##  9 ">9000" 
## 10 "5'7\"" 
## # i 190 more rows
\end{verbatim}

Après analyse, on distingue les problème suivant:

\begin{enumerate}
\def\labelenumi{\arabic{enumi}.}
\tightlist
\item
  x'y ou x' y' ' ou x'y'\,' ou x reprentes le foot et y le inches
\item
  x.y. ou x,y de la même manière x represente le foot et y le inches
\item
  ceux qui sont representé en cm
\item
  et bien evidement d'autre entre
\end{enumerate}

Tout d'abord on va commencer par ces trois premier ensuite attacker le
reste. On va utilisé les expression régulier pour effectuer ces text.

\hypertarget{les-expressions-ruxe9guliuxe8re}{%
\paragraph{Les Expressions
régulière}\label{les-expressions-ruxe9guliuxe8re}}

Une expresion regulière est une description d'un modèle d'un ensemble
chaine de caractère utilisée pour effectuer des recherches dans un text.
Si on demande á un utilisateur d'entrer son adresse mail, on peux
définir un modèl d'adress mail avec qui on verifie si l'adress rentré
correspond bien au norme. Voic quelques exemples d'expression régulière
en r:

\begin{itemize}
\item
  \textbackslash\textbackslash d signifie un chiffre quelconques
\item
  {[}{]} ceci represente un classe de charactère ei: {[}1-5{]} peux
  correspondre a 1,2,3,4 ou 5
\item
  Anchors: permet de definir la debut et la fin du modèle. \^{} debut \$
  fin
\item
  quantificateurs: permet de definir un nombre de charactère \{1,3,4\}
  1, 3 ou c4 charactère
\item
  \s represente les espaces
\item
  +, *, ? represente respectivement 1 ou plus, 0 ou plus, 1 et 1 seul
  charactère
\item
  \^{} á l'interieur d'un chrochet signifie non ex : {[}\^{}a-zA-Z{]}
  pas d'alphabet majuscule ou miniscule
\item
  Pour echapper les charactères spéciaux on utilisie \textbackslash{}
  ex: ``\^{}\textbackslash\textbackslash d\textbackslash{}''\$''
\end{itemize}

On a une idée des expression régulier en r, on va parcourir les
différent modèle de problème existant

\begin{Shaded}
\begin{Highlighting}[]
\NormalTok{ model }\OtherTok{\textless{}{-}} \StringTok{"\^{}[4{-}7]\textquotesingle{}}\SpecialCharTok{\textbackslash{}\textbackslash{}}\StringTok{d\{1,2\}}\SpecialCharTok{\textbackslash{}"}\StringTok{$"} \CommentTok{\# commence par un chiffre entre 4 et 7 suivie de \textquotesingle{} et d\textquotesingle{}un ou deux chiffre et "}
 \FunctionTok{str\_subset}\NormalTok{(problematiques, model)}
\end{Highlighting}
\end{Shaded}

\begin{verbatim}
##  [1] "5'7\""  "5'3\""  "5'5\""  "5'2\""  "5'3\""  "5'2\""  "5'8\""  "5'11\""
##  [9] "5'7\""  "6'1\""  "5'8\""  "6'3\""  "5'7\""  "6'4\""
\end{verbatim}

\begin{Shaded}
\begin{Highlighting}[]
\FunctionTok{class}\NormalTok{(problematiques)}
\end{Highlighting}
\end{Shaded}

\begin{verbatim}
## [1] "character"
\end{verbatim}

\begin{Shaded}
\begin{Highlighting}[]
\FunctionTok{str\_subset}\NormalTok{(problematiques, }\StringTok{"inches"}\NormalTok{)}
\end{Highlighting}
\end{Shaded}

\begin{verbatim}
## [1] "5 feet and 8.11 inches" "Five foot eight inches" "5 feet 7inches"        
## [4] "5ft 9 inches"           "5 ft 9 inches"          "5 feet 6 inches"
\end{verbatim}

\begin{Shaded}
\begin{Highlighting}[]
\FunctionTok{str\_subset}\NormalTok{(problematiques, }\StringTok{"\textquotesingle{}\textquotesingle{}"}\NormalTok{)}
\end{Highlighting}
\end{Shaded}

\begin{verbatim}
##  [1] "5'9''"   "5'10''"  "5'10''"  "5'3''"   "5'7''"   "5'6''"   "5'7.5''"
##  [8] "5'7.5''" "5'10''"  "5'11''"  "5'10''"  "5'5''"
\end{verbatim}

\begin{Shaded}
\begin{Highlighting}[]
\FunctionTok{str\_subset}\NormalTok{(problematiques, }\StringTok{"}\SpecialCharTok{\textbackslash{}\textbackslash{}}\StringTok{d\{1\}}\SpecialCharTok{\textbackslash{}\textbackslash{}}\StringTok{.}\SpecialCharTok{\textbackslash{}\textbackslash{}}\StringTok{d\{1,2\}"}\NormalTok{)}
\end{Highlighting}
\end{Shaded}

\begin{verbatim}
##  [1] "5.3"                    "5 feet and 8.11 inches" "5.25"                  
##  [4] "5.5"                    "6.5"                    "103.2"                 
##  [7] "5.8"                    "5.6"                    "5.9"                   
## [10] "5.5"                    "6.2"                    "6.2"                   
## [13] "5.8"                    "5.1"                    "5.11"                  
## [16] "5.75"                   "5.4"                    "6.1"                   
## [19] "5.6"                    "5.6"                    "0.7"                   
## [22] "5.4"                    "5.9"                    "5.6"                   
## [25] "5.6"                    "5.5"                    "5.2"                   
## [28] "5.5"                    "5.5"                    "6.5"                   
## [31] "5.11"                   "5.5"                    "5'7.5''"               
## [34] "5'7.5''"                "5' 7.78\""              "6.7"                   
## [37] "5.1"                    "5.6"                    "5.5"                   
## [40] "5.2"                    "5.6"                    "5.7"                   
## [43] "5.9"                    "6.5"                    "5.11"                  
## [46] "1.6"                    "5.7"                    "5.5"                   
## [49] "1.7"                    "5.8"                    "5.8"                   
## [52] "5.1"                    "5.11"                   "5.7"                   
## [55] "5.9"                    "5.2"                    "5.5"                   
## [58] "5.51"                   "5.8"                    "5.7"                   
## [61] "6.1"                    "5.69"                   "5.7"                   
## [64] "5.25"                   "5.5"                    "5.1"                   
## [67] "6.3"                    "5.5"                    "5.7"                   
## [70] "5.57"                   "5.7"
\end{verbatim}

\begin{Shaded}
\begin{Highlighting}[]
\NormalTok{pattern }\OtherTok{\textless{}{-}} \StringTok{"\^{}[4{-}7]}\SpecialCharTok{\textbackslash{}\textbackslash{}}\StringTok{s*\textquotesingle{}}\SpecialCharTok{\textbackslash{}\textbackslash{}}\StringTok{s*}\SpecialCharTok{\textbackslash{}\textbackslash{}}\StringTok{d\{1,2\}$"}
\FunctionTok{str\_subset}\NormalTok{(problematiques, pattern)}
\end{Highlighting}
\end{Shaded}

\begin{verbatim}
##  [1] "5'7"   "5'11"  "5' 10" "5'3"   "5'12"  "5'11"  "5'4"   "5'5"   "5'8"  
## [10] "5'6"   "5'4"   "5'5"   "5'7"   "5'6"   "5'7"   "5'8"   "5'10"  "5'10" 
## [19] "5'2"   "5'11"  "5'8"   "5'9"   "5'4"   "5'6"   "5'6"
\end{verbatim}

\begin{Shaded}
\begin{Highlighting}[]
\NormalTok{model }\OtherTok{\textless{}{-}} \StringTok{"\^{}[4{-}7]}\SpecialCharTok{\textbackslash{}\textbackslash{}}\StringTok{s*\textquotesingle{}}\SpecialCharTok{\textbackslash{}\textbackslash{}}\StringTok{s*}\SpecialCharTok{\textbackslash{}\textbackslash{}}\StringTok{d\{1,2\}$"} \CommentTok{\# par exemple 5\textquotesingle{}10 correspond}
\end{Highlighting}
\end{Shaded}

\begin{Shaded}
\begin{Highlighting}[]
\NormalTok{problematiques }\SpecialCharTok{\%\textgreater{}\%} \FunctionTok{str\_replace}\NormalTok{(}\StringTok{"feet|ft|foot"}\NormalTok{, }\StringTok{"\textquotesingle{}"}\NormalTok{) }\SpecialCharTok{\%\textgreater{}\%} 
  \FunctionTok{str\_replace}\NormalTok{(}\StringTok{"inches|in|\textquotesingle{}\textquotesingle{}|}\SpecialCharTok{\textbackslash{}"}\StringTok{"}\NormalTok{, }\StringTok{""}\NormalTok{) }\SpecialCharTok{\%\textgreater{}\%} 
  \FunctionTok{str\_detect}\NormalTok{(}\AttributeTok{pattern =}\NormalTok{ model) }\SpecialCharTok{\%\textgreater{}\%} \FunctionTok{sum}\NormalTok{()}
\end{Highlighting}
\end{Shaded}

\begin{verbatim}
## [1] 53
\end{verbatim}

On ecris deux fonction:

\begin{itemize}
\item
  Une fonction qui reconvertie le format des données
\item
  Une fonction qui convertis les nombres qui sont écris en lettres en
  chiffres
\end{itemize}

\begin{Shaded}
\begin{Highlighting}[]
\NormalTok{convert\_format }\OtherTok{\textless{}{-}} \ControlFlowTok{function}\NormalTok{(s)\{}
\NormalTok{   s }\SpecialCharTok{\%\textgreater{}\%} 
    \FunctionTok{str\_replace}\NormalTok{(}\StringTok{"feet|foot|ft"}\NormalTok{, }\StringTok{"\textquotesingle{}"}\NormalTok{) }\SpecialCharTok{\%\textgreater{}\%}  
    \FunctionTok{str\_replace\_all}\NormalTok{(}\StringTok{"inches|in|\textquotesingle{}\textquotesingle{}|}\SpecialCharTok{\textbackslash{}"}\StringTok{|cm|and"}\NormalTok{, }\StringTok{""}\NormalTok{) }\SpecialCharTok{\%\textgreater{}\%} 
    \FunctionTok{str\_replace}\NormalTok{(}\StringTok{"\^{}([4{-}7])}\SpecialCharTok{\textbackslash{}\textbackslash{}}\StringTok{s*[,}\SpecialCharTok{\textbackslash{}\textbackslash{}}\StringTok{.}\SpecialCharTok{\textbackslash{}\textbackslash{}}\StringTok{s+]}\SpecialCharTok{\textbackslash{}\textbackslash{}}\StringTok{s*(}\SpecialCharTok{\textbackslash{}\textbackslash{}}\StringTok{d*)$"}\NormalTok{, }\StringTok{"}\SpecialCharTok{\textbackslash{}\textbackslash{}}\StringTok{1\textquotesingle{}}\SpecialCharTok{\textbackslash{}\textbackslash{}}\StringTok{2"}\NormalTok{) }\SpecialCharTok{\%\textgreater{}\%} 
    \FunctionTok{str\_replace}\NormalTok{(}\StringTok{"\^{}([56])\textquotesingle{}?$"}\NormalTok{, }\StringTok{"}\SpecialCharTok{\textbackslash{}\textbackslash{}}\StringTok{1\textquotesingle{}0"}\NormalTok{) }\SpecialCharTok{\%\textgreater{}\%} 
    \FunctionTok{str\_replace}\NormalTok{(}\StringTok{"\^{}([12])}\SpecialCharTok{\textbackslash{}\textbackslash{}}\StringTok{s*,}\SpecialCharTok{\textbackslash{}\textbackslash{}}\StringTok{s*(}\SpecialCharTok{\textbackslash{}\textbackslash{}}\StringTok{d*)$"}\NormalTok{, }\StringTok{"}\SpecialCharTok{\textbackslash{}\textbackslash{}}\StringTok{1}\SpecialCharTok{\textbackslash{}\textbackslash{}}\StringTok{.}\SpecialCharTok{\textbackslash{}\textbackslash{}}\StringTok{2"}\NormalTok{) }\SpecialCharTok{\%\textgreater{}\%} 
    \FunctionTok{str\_trim}\NormalTok{()}
\NormalTok{\}}

\CommentTok{\# convert word to number}
\NormalTok{words\_to\_numbers }\OtherTok{\textless{}{-}} \ControlFlowTok{function}\NormalTok{(s)\{ }\FunctionTok{str\_to\_lower}\NormalTok{(s) }\SpecialCharTok{\%\textgreater{}\%}
    \FunctionTok{str\_replace\_all}\NormalTok{(}\StringTok{"zero"}\NormalTok{, }\StringTok{"0"}\NormalTok{) }\SpecialCharTok{\%\textgreater{}\%}
    \FunctionTok{str\_replace\_all}\NormalTok{(}\StringTok{"one"}\NormalTok{, }\StringTok{"1"}\NormalTok{) }\SpecialCharTok{\%\textgreater{}\%} 
    \FunctionTok{str\_replace\_all}\NormalTok{(}\StringTok{"two"}\NormalTok{, }\StringTok{"2"}\NormalTok{) }\SpecialCharTok{\%\textgreater{}\%}
    \FunctionTok{str\_replace\_all}\NormalTok{(}\StringTok{"three"}\NormalTok{, }\StringTok{"3"}\NormalTok{) }\SpecialCharTok{\%\textgreater{}\%}
    \FunctionTok{str\_replace\_all}\NormalTok{(}\StringTok{"four"}\NormalTok{, }\StringTok{"4"}\NormalTok{) }\SpecialCharTok{\%\textgreater{}\%} 
    \FunctionTok{str\_replace\_all}\NormalTok{(}\StringTok{"five"}\NormalTok{, }\StringTok{"5"}\NormalTok{) }\SpecialCharTok{\%\textgreater{}\%} 
    \FunctionTok{str\_replace\_all}\NormalTok{(}\StringTok{"six"}\NormalTok{, }\StringTok{"6"}\NormalTok{) }\SpecialCharTok{\%\textgreater{}\%} 
    \FunctionTok{str\_replace\_all}\NormalTok{(}\StringTok{"seven"}\NormalTok{, }\StringTok{"7"}\NormalTok{) }\SpecialCharTok{\%\textgreater{}\%} 
    \FunctionTok{str\_replace\_all}\NormalTok{(}\StringTok{"eight"}\NormalTok{, }\StringTok{"8"}\NormalTok{) }\SpecialCharTok{\%\textgreater{}\%}
    \FunctionTok{str\_replace\_all}\NormalTok{(}\StringTok{"nine"}\NormalTok{, }\StringTok{"9"}\NormalTok{) }\SpecialCharTok{\%\textgreater{}\%}
    \FunctionTok{str\_replace\_all}\NormalTok{(}\StringTok{"ten"}\NormalTok{, }\StringTok{"10"}\NormalTok{) }\SpecialCharTok{\%\textgreater{}\%} 
    \FunctionTok{str\_replace\_all}\NormalTok{(}\StringTok{"eleven"}\NormalTok{, }\StringTok{"11"}\NormalTok{)}
\NormalTok{\}}
\end{Highlighting}
\end{Shaded}

Maintenant on fait tous en une seul bloques

\begin{Shaded}
\begin{Highlighting}[]
\NormalTok{pattern }\OtherTok{\textless{}{-}} \StringTok{"\^{}([4{-}7])}\SpecialCharTok{\textbackslash{}\textbackslash{}}\StringTok{s*\textquotesingle{}}\SpecialCharTok{\textbackslash{}\textbackslash{}}\StringTok{s*(}\SpecialCharTok{\textbackslash{}\textbackslash{}}\StringTok{d+}\SpecialCharTok{\textbackslash{}\textbackslash{}}\StringTok{.?}\SpecialCharTok{\textbackslash{}\textbackslash{}}\StringTok{d*)$"}
\NormalTok{smallest }\OtherTok{\textless{}{-}} \DecValTok{50}
\NormalTok{tallest }\OtherTok{\textless{}{-}} \DecValTok{84}
\NormalTok{new\_heights }\OtherTok{\textless{}{-}}\NormalTok{ reported\_heights }\SpecialCharTok{\%\textgreater{}\%}
\FunctionTok{mutate}\NormalTok{(}\AttributeTok{original =}\NormalTok{ height, }\AttributeTok{height =} \FunctionTok{words\_to\_numbers}\NormalTok{(height) }\SpecialCharTok{\%\textgreater{}\%} \FunctionTok{convert\_format}\NormalTok{()) }\SpecialCharTok{\%\textgreater{}\%} \FunctionTok{extract}\NormalTok{(height, }\FunctionTok{c}\NormalTok{(}\StringTok{"feet"}\NormalTok{, }\StringTok{"inches"}\NormalTok{), }\AttributeTok{regex =}\NormalTok{ pattern, }\AttributeTok{remove =} \ConstantTok{FALSE}\NormalTok{) }\SpecialCharTok{\%\textgreater{}\%} \FunctionTok{mutate\_at}\NormalTok{(}\FunctionTok{c}\NormalTok{(}\StringTok{"height"}\NormalTok{, }\StringTok{"feet"}\NormalTok{, }\StringTok{"inches"}\NormalTok{), as.numeric) }\SpecialCharTok{\%\textgreater{}\%}
\FunctionTok{mutate}\NormalTok{(}\AttributeTok{guess =} \DecValTok{12}\SpecialCharTok{*}\NormalTok{feet }\SpecialCharTok{+}\NormalTok{ inches) }\SpecialCharTok{\%\textgreater{}\%}
\FunctionTok{mutate}\NormalTok{(}\AttributeTok{height =} \FunctionTok{case\_when}\NormalTok{(}
\SpecialCharTok{!}\FunctionTok{is.na}\NormalTok{(height) }\SpecialCharTok{\&} \FunctionTok{between}\NormalTok{(height, smallest, tallest) }\SpecialCharTok{\textasciitilde{}}\NormalTok{ height, }
\SpecialCharTok{!}\FunctionTok{is.na}\NormalTok{(height) }\SpecialCharTok{\&} \FunctionTok{between}\NormalTok{(height}\SpecialCharTok{/}\FloatTok{2.54}\NormalTok{, smallest, tallest) }\SpecialCharTok{\textasciitilde{}}\NormalTok{ height}\SpecialCharTok{/}\FloatTok{2.54}\NormalTok{, }
\SpecialCharTok{!}\FunctionTok{is.na}\NormalTok{(height) }\SpecialCharTok{\&} \FunctionTok{between}\NormalTok{(height}\SpecialCharTok{*}\DecValTok{100}\SpecialCharTok{/}\FloatTok{2.54}\NormalTok{, smallest, tallest) }\SpecialCharTok{\textasciitilde{}}\NormalTok{ height}\SpecialCharTok{*}\DecValTok{100}\SpecialCharTok{/}\FloatTok{2.54}\NormalTok{,}
\SpecialCharTok{!}\FunctionTok{is.na}\NormalTok{(guess) }\SpecialCharTok{\&}\NormalTok{ inches }\SpecialCharTok{\textless{}} \DecValTok{12} \SpecialCharTok{\&} \FunctionTok{between}\NormalTok{(guess, smallest, tallest) }\SpecialCharTok{\textasciitilde{}}\NormalTok{ guess, }
\ConstantTok{TRUE} \SpecialCharTok{\textasciitilde{}} \FunctionTok{as.numeric}\NormalTok{(}\ConstantTok{NA}\NormalTok{))) }\SpecialCharTok{\%\textgreater{}\%}
\FunctionTok{select}\NormalTok{(}\SpecialCharTok{{-}}\NormalTok{guess)}
\end{Highlighting}
\end{Shaded}

\begin{verbatim}
## Warning: There was 1 warning in `mutate()`.
## i In argument: `height = .Primitive("as.double")(height)`.
## Caused by warning:
## ! NAs introduced by coercion
\end{verbatim}

\begin{Shaded}
\begin{Highlighting}[]
\NormalTok{new\_heights }\SpecialCharTok{\%\textgreater{}\%} \FunctionTok{filter}\NormalTok{(}\FunctionTok{not\_inches\_or\_cm}\NormalTok{(original)) }\SpecialCharTok{\%\textgreater{}\%}
  \FunctionTok{select}\NormalTok{(original, height) }\SpecialCharTok{\%\textgreater{}\%} 
  \FunctionTok{arrange}\NormalTok{(height)}
\end{Highlighting}
\end{Shaded}

\begin{verbatim}
##                   original   height
## 1                        5 60.00000
## 2                        5 60.00000
## 3                        5 60.00000
## 4                        5 60.00000
## 5                      5.1 61.00000
## 6                      5.1 61.00000
## 7                      5.1 61.00000
## 8                      5.1 61.00000
## 9                     5'2" 62.00000
## 10                     5.2 62.00000
## 11                    5'2" 62.00000
## 12                     5.2 62.00000
## 13                     5.2 62.00000
## 14                     5'2 62.00000
## 15                     1.6 62.99213
## 16                     5.3 63.00000
## 17                    5'3" 63.00000
## 18                     5,3 63.00000
## 19                     5'3 63.00000
## 20                   5'3'' 63.00000
## 21                    5'3" 63.00000
## 22                   5' 4" 64.00000
## 23                     5,4 64.00000
## 24                     5.4 64.00000
## 25                     5.4 64.00000
## 26                     5'4 64.00000
## 27                     5'4 64.00000
## 28                     5'4 64.00000
## 29                   165cm 64.96063
## 30                     5.5 65.00000
## 31                     5.5 65.00000
## 32                    5'5" 65.00000
## 33                     5.5 65.00000
## 34                     5.5 65.00000
## 35                     5.5 65.00000
## 36                     5.5 65.00000
## 37                     5.5 65.00000
## 38                     5'5 65.00000
## 39                     5.5 65.00000
## 40                     5'5 65.00000
## 41                     5.5 65.00000
## 42                     5.5 65.00000
## 43                   5'5'' 65.00000
## 44                     5.5 65.00000
## 45                     5.6 66.00000
## 46                     5.6 66.00000
## 47                     5.6 66.00000
## 48                     5.6 66.00000
## 49                     5.6 66.00000
## 50                   5'6'' 66.00000
## 51                     5.6 66.00000
## 52                     5'6 66.00000
## 53                     5.6 66.00000
## 54                     5'6 66.00000
## 55         5 feet 6 inches 66.00000
## 56                     5'6 66.00000
## 57                     5'6 66.00000
## 58                    1,70 66.92913
## 59                     1.7 66.92913
## 60                  170 cm 66.92913
## 61                     5'7 67.00000
## 62                    5'7" 67.00000
## 63                   5'7'' 67.00000
## 64          5 feet 7inches 67.00000
## 65                     5.7 67.00000
## 66                     5.7 67.00000
## 67                     5.7 67.00000
## 68                     5'7 67.00000
## 69                    5'7" 67.00000
## 70                     5'7 67.00000
## 71                   5' 7" 67.00000
## 72                     5.7 67.00000
## 73                     5.7 67.00000
## 74                    5'7" 67.00000
## 75                     5.7 67.00000
## 76                     5.7 67.00000
## 77                 5'7.5'' 67.50000
## 78                 5'7.5'' 67.50000
## 79                5' 7.78" 67.78000
## 80                     5.8 68.00000
## 81  Five foot eight inches 68.00000
## 82                     5.8 68.00000
## 83                     5,8 68.00000
## 84                     5'8 68.00000
## 85                     5.8 68.00000
## 86                    5'8" 68.00000
## 87                     5.8 68.00000
## 88                     5'8 68.00000
## 89                     5.8 68.00000
## 90                    5'8" 68.00000
## 91                     5'8 68.00000
## 92  5 feet and 8.11 inches 68.11000
## 93                   5'9'' 69.00000
## 94                     5.9 69.00000
## 95                     5.9 69.00000
## 96                     5.9 69.00000
## 97                     5.9 69.00000
## 98                     69" 69.00000
## 99            5ft 9 inches 69.00000
## 100          5 ft 9 inches 69.00000
## 101                    5'9 69.00000
## 102                 5'10'' 70.00000
## 103                  5' 10 70.00000
## 104                 5'10'' 70.00000
## 105                 5'10'' 70.00000
## 106                   5'10 70.00000
## 107                   5'10 70.00000
## 108                 5'10'' 70.00000
## 109                   5'11 71.00000
## 110                   5.11 71.00000
## 111                   5'11 71.00000
## 112                   5.11 71.00000
## 113                   5.11 71.00000
## 114                  5 .11 71.00000
## 115                   5 11 71.00000
## 116                   5.11 71.00000
## 117                  5'11" 71.00000
## 118                 5' 11" 71.00000
## 119                   5'11 71.00000
## 120                 5'11'' 71.00000
## 121                      6 72.00000
## 122                      6 72.00000
## 123                      6 72.00000
## 124                     6' 72.00000
## 125                      6 72.00000
## 126                      6 72.00000
## 127                      6 72.00000
## 128                      6 72.00000
## 129                      6 72.00000
## 130                      6 72.00000
## 131                      6 72.00000
## 132                      6 72.00000
## 133                      6 72.00000
## 134                      6 72.00000
## 135                      6 72.00000
## 136                      6 72.00000
## 137                      6 72.00000
## 138                      6 72.00000
## 139                      6 72.00000
## 140                      6 72.00000
## 141                    6.1 73.00000
## 142                   6'1" 73.00000
## 143                    6.1 73.00000
## 144                    6.2 74.00000
## 145                    6.2 74.00000
## 146                   6'3" 75.00000
## 147                    6.3 75.00000
## 148                   6 04 76.00000
## 149                   6'4" 76.00000
## 150                    6.5 77.00000
## 151                    6.5 77.00000
## 152                    6.5 77.00000
## 153                      2 78.74016
## 154                    6.7 79.00000
## 155                    6,8 80.00000
## 156                    511       NA
## 157                  >9000       NA
## 158                   5.25       NA
## 159                  11111       NA
## 160                  103.2       NA
## 161                     19       NA
## 162                    300       NA
## 163                   5.75       NA
## 164                      7       NA
## 165                    214       NA
## 166                    0.7       NA
## 167                   5'12       NA
## 168                   2'33       NA
## 169                    612       NA
## 170                     87       NA
## 171                    111       NA
## 172                     12       NA
## 173                    yyy       NA
## 174                     89       NA
## 175                     34       NA
## 176                     25       NA
## 177                     22       NA
## 178                    684       NA
## 179                      1       NA
## 180                      1       NA
## 181                   6*12       NA
## 182                     87       NA
## 183                    120       NA
## 184                    120       NA
## 185                     23       NA
## 186                   5.51       NA
## 187                   5.69       NA
## 188                     86       NA
## 189                708,661       NA
## 190                   5.25       NA
## 191                649,606       NA
## 192                  10000       NA
## 193                      1       NA
## 194                728,346       NA
## 195                      0       NA
## 196                    100       NA
## 197                   5.57       NA
## 198                     88       NA
## 199              7,283,465       NA
## 200                     34       NA
\end{verbatim}

\begin{Shaded}
\begin{Highlighting}[]
\NormalTok{new\_heights[}\FunctionTok{c}\NormalTok{(}\DecValTok{1}\SpecialCharTok{:}\DecValTok{10}\NormalTok{, }\DecValTok{30}\SpecialCharTok{:}\DecValTok{80}\NormalTok{, }\DecValTok{100}\SpecialCharTok{:}\DecValTok{150}\NormalTok{),]}
\end{Highlighting}
\end{Shaded}

\begin{verbatim}
##              time_stamp    sex   height feet inches original
## 1   2014-09-02 13:40:36   Male 75.00000   NA     NA       75
## 2   2014-09-02 13:46:59   Male 70.00000   NA     NA       70
## 3   2014-09-02 13:59:20   Male 68.00000   NA     NA       68
## 4   2014-09-02 14:51:53   Male 74.00000   NA     NA       74
## 5   2014-09-02 15:16:15   Male 61.00000   NA     NA       61
## 6   2014-09-02 15:16:16 Female 65.00000   NA     NA       65
## 7   2014-09-02 15:16:19 Female 66.00000   NA     NA       66
## 8   2014-09-02 15:16:21 Female 62.00000   NA     NA       62
## 9   2014-09-02 15:16:21 Female 66.00000   NA     NA       66
## 10  2014-09-02 15:16:22   Male 67.00000   NA     NA       67
## 30  2014-09-02 15:16:30   Male 72.00000   NA     NA       72
## 31  2014-09-02 15:16:30   Male 75.00000   NA     NA       75
## 32  2014-09-02 15:16:30   Male 71.00000   NA     NA       71
## 33  2014-09-02 15:16:31 Female 67.00000   NA     NA       67
## 34  2014-09-02 15:16:31 Female 66.00000   NA     NA       66
## 35  2014-09-02 15:16:31 Female 67.00000   NA     NA       67
## 36  2014-09-02 15:16:31   Male 69.00000   NA     NA       69
## 37  2014-09-02 15:16:31   Male 68.00000   NA     NA       68
## 38  2014-09-02 15:16:31 Female 66.75000   NA     NA    66.75
## 39  2014-09-02 15:16:31   Male 72.00000   NA     NA       72
## 40  2014-09-02 15:16:32 Female 63.00000    5      3      5.3
## 41  2014-09-02 15:16:33   Male 69.00000   NA     NA       69
## 42  2014-09-02 15:16:33   Male 68.00000   NA     NA       68
## 43  2014-09-02 15:16:33 Female 63.00000   NA     NA       63
## 44  2014-09-02 15:16:33   Male 60.00000   NA     NA       60
## 45  2014-09-02 15:16:33   Male 73.00000   NA     NA       73
## 46  2014-09-02 15:16:33   Male 74.00000   NA     NA       74
## 47  2014-09-02 15:16:33   Male 74.00000   NA     NA       74
## 48  2014-09-02 15:16:34   Male 66.00000   NA     NA       66
## 49  2014-09-02 15:16:34   Male 68.00000   NA     NA       68
## 50  2014-09-02 15:16:34   Male 73.00000   NA     NA       73
## 51  2014-09-02 15:16:35   Male 70.00000   NA     NA       70
## 52  2014-09-02 15:16:35   Male 68.00000   NA     NA       68
## 53  2014-09-02 15:16:36   Male 73.00000   NA     NA       73
## 54  2014-09-02 15:16:37   Male 70.50000   NA     NA     70.5
## 55  2014-09-02 15:16:37 Female 64.96063   NA     NA    165cm
## 56  2014-09-02 15:16:37   Male 71.00000   NA     NA       71
## 57  2014-09-02 15:16:37   Male 70.00000   NA     NA       70
## 58  2014-09-02 15:16:38   Male 67.00000   NA     NA       67
## 59  2014-09-02 15:16:38   Male 69.00000   NA     NA       69
## 60  2014-09-02 15:16:38   Male 67.00000   NA     NA       67
## 61  2014-09-02 15:16:39   Male 69.00000   NA     NA       69
## 62  2014-09-02 15:16:39   Male 73.00000   NA     NA       73
## 63  2014-09-02 15:16:40   Male 74.00000   NA     NA       74
## 64  2014-09-02 15:16:41   Male 70.00000   NA     NA       70
## 65  2014-09-02 15:16:41   Male 66.00000   NA     NA       66
## 66  2014-09-02 15:16:41   Male       NA   NA     NA      511
## 67  2014-09-02 15:16:41   Male 72.00000   NA     NA       72
## 68  2014-09-02 15:16:41 Female 65.00000   NA     NA       65
## 69  2014-09-02 15:16:42   Male 65.00000   NA     NA       65
## 70  2014-09-02 15:16:42   Male 70.00000   NA     NA       70
## 71  2014-09-02 15:16:42   Male 73.00000   NA     NA       73
## 72  2014-09-02 15:16:42   Male 67.00000   NA     NA       67
## 73  2014-09-02 15:16:43   Male 72.00000   NA     NA       72
## 74  2014-09-02 15:16:43   Male 68.00000   NA     NA       68
## 75  2014-09-02 15:16:43   Male 68.00000   NA     NA       68
## 76  2014-09-02 15:16:43 Female 65.00000   NA     NA       65
## 77  2014-09-02 15:16:43   Male 72.00000   NA     NA       72
## 78  2014-09-02 15:16:44   Male 71.00000   NA     NA       71
## 79  2014-09-02 15:16:44 Female 65.00000   NA     NA       65
## 80  2014-09-02 15:16:45 Female 72.00000   NA     NA       72
## 100 2014-09-02 15:16:50   Male 70.00000   NA     NA       70
## 101 2014-09-02 15:16:50   Male 72.00000   NA     NA       72
## 102 2014-09-02 15:16:50 Female 71.00000   NA     NA       71
## 103 2014-09-02 15:16:51   Male 71.00000   NA     NA       71
## 104 2014-09-02 15:16:51 Female 69.60000   NA     NA     69.6
## 105 2014-09-02 15:16:51 Female 65.00000   NA     NA       65
## 106 2014-09-02 15:16:51   Male 70.00000   NA     NA       70
## 107 2014-09-02 15:16:51 Female 61.00000   NA     NA       61
## 108 2014-09-02 15:16:52 Female 63.00000   NA     NA       63
## 109 2014-09-02 15:16:52   Male 70.00000   NA     NA       70
## 110 2014-09-02 15:16:52   Male 67.00000    5      7      5'7
## 111 2014-09-02 15:16:52 Female 62.00000   NA     NA       62
## 112 2014-09-02 15:16:53   Male 68.00000   NA     NA       68
## 113 2014-09-02 15:16:53   Male 73.00000   NA     NA       73
## 114 2014-09-02 15:16:53 Female 66.50000   NA     NA     66.5
## 115 2014-09-02 15:16:53   Male 69.00000   NA     NA       69
## 116 2014-09-02 15:16:53   Male 74.00000   NA     NA       74
## 117 2014-09-02 15:16:54   Male 71.50000   NA     NA     71.5
## 118 2014-09-02 15:16:54   Male 76.00000   NA     NA       76
## 119 2014-09-02 15:16:54   Male 69.00000   NA     NA       69
## 120 2014-09-02 15:16:54   Male 74.00000   NA     NA       74
## 121 2014-09-02 15:16:55   Male 74.50000   NA     NA     74.5
## 122 2014-09-02 15:16:55   Male 69.00000   NA     NA       69
## 123 2014-09-02 15:16:55 Female 66.00000   NA     NA       66
## 124 2014-09-02 15:16:55 Female 64.00000   NA     NA       64
## 125 2014-09-02 15:16:55   Male 78.00000   NA     NA       78
## 126 2014-09-02 15:16:56   Male       NA   NA     NA    >9000
## 127 2014-09-02 15:16:56   Male 67.00000    5      7     5'7"
## 128 2014-09-02 15:16:56 Female 69.00000   NA     NA       69
## 129 2014-09-02 15:16:57 Female 67.00000   NA     NA       67
## 130 2014-09-02 15:16:58 Female 63.00000   NA     NA       63
## 131 2014-09-02 15:16:58   Male 74.00000   NA     NA       74
## 132 2014-09-02 15:16:59 Female 62.00000   NA     NA       62
## 133 2014-09-02 15:16:59 Female 69.00000   NA     NA       69
## 134 2014-09-02 15:16:59 Female 64.00000   NA     NA       64
## 135 2014-09-02 15:17:01   Male 71.00000   NA     NA       71
## 136 2014-09-02 15:17:02 Female 62.50000   NA     NA     62.5
## 137 2014-09-02 15:17:02   Male 68.00000   NA     NA       68
## 138 2014-09-02 15:17:02 Female 67.00000   NA     NA       67
## 139 2014-09-02 15:17:03   Male 71.00000   NA     NA       71
## 140 2014-09-02 15:17:03   Male 74.00000   NA     NA       74
## 141 2014-09-02 15:17:05   Male 75.00000   NA     NA       75
## 142 2014-09-02 15:17:06 Female 65.00000   NA     NA       65
## 143 2014-09-02 15:17:06   Male 68.00000   NA     NA       68
## 144 2014-09-02 15:17:07 Female 65.00000   NA     NA       65
## 145 2014-09-02 15:17:07 Female 66.00000   NA     NA       66
## 146 2014-09-02 15:17:07   Male 72.00000   NA     NA       72
## 147 2014-09-02 15:17:08   Male 73.00000   NA     NA       73
## 148 2014-09-02 15:17:08   Male 71.00000   NA     NA       71
## 149 2014-09-02 15:17:08   Male 74.00000   NA     NA       74
## 150 2014-09-02 15:17:09 Female 63.00000    5      3     5'3"
\end{verbatim}

Les données ont été bien nétoyé

\hypertarget{conclusion}{%
\subsubsection{Conclusion}\label{conclusion}}

Ce cas d'utilisation a mis montre quelques problèmes qu'on peux
rencontrer quand on procéde les chaînes de caractère en r et comment y
remedier. Pour purifier les chaînes de charactères, il est très
important de maitriser les expression régulier en r qui ne sont qu'un
modèle de recherche dans un text. Ces dernier donne un moyen efficace et
concise de netoyer les chaine de caractère. Le traitement de chaîne de
caractère ne sont pas que les seul aspect de la purification de données,
un autre aspect est le traitement des dates, suppression des doublons,
traitement des valeurs manquantes etc.

\end{document}
